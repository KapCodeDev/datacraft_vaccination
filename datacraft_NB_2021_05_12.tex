% Options for packages loaded elsewhere
\PassOptionsToPackage{unicode}{hyperref}
\PassOptionsToPackage{hyphens}{url}
%
\documentclass[
]{article}
\usepackage{lmodern}
\usepackage{amssymb,amsmath}
\usepackage{ifxetex,ifluatex}
\ifnum 0\ifxetex 1\fi\ifluatex 1\fi=0 % if pdftex
  \usepackage[T1]{fontenc}
  \usepackage[utf8]{inputenc}
  \usepackage{textcomp} % provide euro and other symbols
\else % if luatex or xetex
  \usepackage{unicode-math}
  \defaultfontfeatures{Scale=MatchLowercase}
  \defaultfontfeatures[\rmfamily]{Ligatures=TeX,Scale=1}
\fi
% Use upquote if available, for straight quotes in verbatim environments
\IfFileExists{upquote.sty}{\usepackage{upquote}}{}
\IfFileExists{microtype.sty}{% use microtype if available
  \usepackage[]{microtype}
  \UseMicrotypeSet[protrusion]{basicmath} % disable protrusion for tt fonts
}{}
\makeatletter
\@ifundefined{KOMAClassName}{% if non-KOMA class
  \IfFileExists{parskip.sty}{%
    \usepackage{parskip}
  }{% else
    \setlength{\parindent}{0pt}
    \setlength{\parskip}{6pt plus 2pt minus 1pt}}
}{% if KOMA class
  \KOMAoptions{parskip=half}}
\makeatother
\usepackage{xcolor}
\IfFileExists{xurl.sty}{\usepackage{xurl}}{} % add URL line breaks if available
\IfFileExists{bookmark.sty}{\usepackage{bookmark}}{\usepackage{hyperref}}
\hypersetup{
  pdftitle={ATELIER TWITTER - 12 MAI 2021},
  hidelinks,
  pdfcreator={LaTeX via pandoc}}
\urlstyle{same} % disable monospaced font for URLs
\usepackage[margin=1in]{geometry}
\usepackage{color}
\usepackage{fancyvrb}
\newcommand{\VerbBar}{|}
\newcommand{\VERB}{\Verb[commandchars=\\\{\}]}
\DefineVerbatimEnvironment{Highlighting}{Verbatim}{commandchars=\\\{\}}
% Add ',fontsize=\small' for more characters per line
\usepackage{framed}
\definecolor{shadecolor}{RGB}{248,248,248}
\newenvironment{Shaded}{\begin{snugshade}}{\end{snugshade}}
\newcommand{\AlertTok}[1]{\textcolor[rgb]{0.94,0.16,0.16}{#1}}
\newcommand{\AnnotationTok}[1]{\textcolor[rgb]{0.56,0.35,0.01}{\textbf{\textit{#1}}}}
\newcommand{\AttributeTok}[1]{\textcolor[rgb]{0.77,0.63,0.00}{#1}}
\newcommand{\BaseNTok}[1]{\textcolor[rgb]{0.00,0.00,0.81}{#1}}
\newcommand{\BuiltInTok}[1]{#1}
\newcommand{\CharTok}[1]{\textcolor[rgb]{0.31,0.60,0.02}{#1}}
\newcommand{\CommentTok}[1]{\textcolor[rgb]{0.56,0.35,0.01}{\textit{#1}}}
\newcommand{\CommentVarTok}[1]{\textcolor[rgb]{0.56,0.35,0.01}{\textbf{\textit{#1}}}}
\newcommand{\ConstantTok}[1]{\textcolor[rgb]{0.00,0.00,0.00}{#1}}
\newcommand{\ControlFlowTok}[1]{\textcolor[rgb]{0.13,0.29,0.53}{\textbf{#1}}}
\newcommand{\DataTypeTok}[1]{\textcolor[rgb]{0.13,0.29,0.53}{#1}}
\newcommand{\DecValTok}[1]{\textcolor[rgb]{0.00,0.00,0.81}{#1}}
\newcommand{\DocumentationTok}[1]{\textcolor[rgb]{0.56,0.35,0.01}{\textbf{\textit{#1}}}}
\newcommand{\ErrorTok}[1]{\textcolor[rgb]{0.64,0.00,0.00}{\textbf{#1}}}
\newcommand{\ExtensionTok}[1]{#1}
\newcommand{\FloatTok}[1]{\textcolor[rgb]{0.00,0.00,0.81}{#1}}
\newcommand{\FunctionTok}[1]{\textcolor[rgb]{0.00,0.00,0.00}{#1}}
\newcommand{\ImportTok}[1]{#1}
\newcommand{\InformationTok}[1]{\textcolor[rgb]{0.56,0.35,0.01}{\textbf{\textit{#1}}}}
\newcommand{\KeywordTok}[1]{\textcolor[rgb]{0.13,0.29,0.53}{\textbf{#1}}}
\newcommand{\NormalTok}[1]{#1}
\newcommand{\OperatorTok}[1]{\textcolor[rgb]{0.81,0.36,0.00}{\textbf{#1}}}
\newcommand{\OtherTok}[1]{\textcolor[rgb]{0.56,0.35,0.01}{#1}}
\newcommand{\PreprocessorTok}[1]{\textcolor[rgb]{0.56,0.35,0.01}{\textit{#1}}}
\newcommand{\RegionMarkerTok}[1]{#1}
\newcommand{\SpecialCharTok}[1]{\textcolor[rgb]{0.00,0.00,0.00}{#1}}
\newcommand{\SpecialStringTok}[1]{\textcolor[rgb]{0.31,0.60,0.02}{#1}}
\newcommand{\StringTok}[1]{\textcolor[rgb]{0.31,0.60,0.02}{#1}}
\newcommand{\VariableTok}[1]{\textcolor[rgb]{0.00,0.00,0.00}{#1}}
\newcommand{\VerbatimStringTok}[1]{\textcolor[rgb]{0.31,0.60,0.02}{#1}}
\newcommand{\WarningTok}[1]{\textcolor[rgb]{0.56,0.35,0.01}{\textbf{\textit{#1}}}}
\usepackage{graphicx,grffile}
\makeatletter
\def\maxwidth{\ifdim\Gin@nat@width>\linewidth\linewidth\else\Gin@nat@width\fi}
\def\maxheight{\ifdim\Gin@nat@height>\textheight\textheight\else\Gin@nat@height\fi}
\makeatother
% Scale images if necessary, so that they will not overflow the page
% margins by default, and it is still possible to overwrite the defaults
% using explicit options in \includegraphics[width, height, ...]{}
\setkeys{Gin}{width=\maxwidth,height=\maxheight,keepaspectratio}
% Set default figure placement to htbp
\makeatletter
\def\fps@figure{htbp}
\makeatother
\setlength{\emergencystretch}{3em} % prevent overfull lines
\providecommand{\tightlist}{%
  \setlength{\itemsep}{0pt}\setlength{\parskip}{0pt}}
\setcounter{secnumdepth}{-\maxdimen} % remove section numbering

\title{ATELIER TWITTER - 12 MAI 2021}
\author{}
\date{\vspace{-2.5em}}

\begin{document}
\maketitle

Load packages :

\begin{Shaded}
\begin{Highlighting}[]
\KeywordTok{require}\NormalTok{(ggplot2)}
\end{Highlighting}
\end{Shaded}

\begin{verbatim}
## Loading required package: ggplot2
\end{verbatim}

\begin{Shaded}
\begin{Highlighting}[]
\KeywordTok{require}\NormalTok{(scales)}
\end{Highlighting}
\end{Shaded}

\begin{verbatim}
## Loading required package: scales
\end{verbatim}

\begin{Shaded}
\begin{Highlighting}[]
\KeywordTok{require}\NormalTok{(dplyr)}
\end{Highlighting}
\end{Shaded}

\begin{verbatim}
## Loading required package: dplyr
\end{verbatim}

\begin{verbatim}
## 
## Attaching package: 'dplyr'
\end{verbatim}

\begin{verbatim}
## The following objects are masked from 'package:stats':
## 
##     filter, lag
\end{verbatim}

\begin{verbatim}
## The following objects are masked from 'package:base':
## 
##     intersect, setdiff, setequal, union
\end{verbatim}

\begin{Shaded}
\begin{Highlighting}[]
\KeywordTok{require}\NormalTok{(tidytext)}
\end{Highlighting}
\end{Shaded}

\begin{verbatim}
## Loading required package: tidytext
\end{verbatim}

\begin{Shaded}
\begin{Highlighting}[]
\KeywordTok{require}\NormalTok{(tm)}
\end{Highlighting}
\end{Shaded}

\begin{verbatim}
## Loading required package: tm
\end{verbatim}

\begin{verbatim}
## Loading required package: NLP
\end{verbatim}

\begin{verbatim}
## 
## Attaching package: 'NLP'
\end{verbatim}

\begin{verbatim}
## The following object is masked from 'package:ggplot2':
## 
##     annotate
\end{verbatim}

\begin{Shaded}
\begin{Highlighting}[]
\KeywordTok{require}\NormalTok{(stringr)}
\end{Highlighting}
\end{Shaded}

\begin{verbatim}
## Loading required package: stringr
\end{verbatim}

\begin{Shaded}
\begin{Highlighting}[]
\KeywordTok{require}\NormalTok{(BTM)}
\end{Highlighting}
\end{Shaded}

\begin{verbatim}
## Loading required package: BTM
\end{verbatim}

\begin{Shaded}
\begin{Highlighting}[]
\KeywordTok{library}\NormalTok{(udpipe)}
\KeywordTok{require}\NormalTok{(rlist)}
\end{Highlighting}
\end{Shaded}

\begin{verbatim}
## Loading required package: rlist
\end{verbatim}

Load the dataset

\begin{Shaded}
\begin{Highlighting}[]
\NormalTok{path =}\StringTok{ ""}
\KeywordTok{load}\NormalTok{(}\DataTypeTok{file =} \KeywordTok{paste0}\NormalTok{(path,}\StringTok{"datacraft_data_vaccination_5g_2021_05_11.RData"}\NormalTok{))}
\KeywordTok{dim}\NormalTok{(dataset_clean)}
\end{Highlighting}
\end{Shaded}

\begin{verbatim}
## [1] 193209     31
\end{verbatim}

Create time series

\begin{Shaded}
\begin{Highlighting}[]
\NormalTok{daily_TS <-}\StringTok{ }\KeywordTok{data.frame}\NormalTok{(}\KeywordTok{table}\NormalTok{(}\KeywordTok{substr}\NormalTok{(dataset_clean}\OperatorTok{$}\NormalTok{tweet_date, }\DataTypeTok{start =} \DecValTok{1}\NormalTok{,}\DataTypeTok{stop =} \DecValTok{10}\NormalTok{)))}
\NormalTok{daily_TS}\OperatorTok{$}\NormalTok{Var1 <-}\StringTok{ }\KeywordTok{as.Date}\NormalTok{(daily_TS}\OperatorTok{$}\NormalTok{Var1)}
\KeywordTok{colnames}\NormalTok{(daily_TS) <-}\StringTok{ }\KeywordTok{c}\NormalTok{(}\StringTok{"dates"}\NormalTok{,}\StringTok{"N"}\NormalTok{)}
\NormalTok{p <-}\StringTok{ }\KeywordTok{ggplot}\NormalTok{(daily_TS, }\KeywordTok{aes}\NormalTok{(dates, N)) }\OperatorTok{+}
\StringTok{  }\KeywordTok{geom_bar}\NormalTok{(}\DataTypeTok{stat=}\StringTok{"identity"}\NormalTok{, }\DataTypeTok{na.rm =} \OtherTok{TRUE}\NormalTok{)}\OperatorTok{+}
\StringTok{  }\KeywordTok{scale_x_date}\NormalTok{(}\DataTypeTok{labels=}\KeywordTok{date_format}\NormalTok{ (}\StringTok{"%b %y"}\NormalTok{), }\DataTypeTok{breaks=}\KeywordTok{date_breaks}\NormalTok{(}\StringTok{"1 month"}\NormalTok{)) }\OperatorTok{+}\StringTok{ }\KeywordTok{theme_minimal}\NormalTok{()}
\KeywordTok{print}\NormalTok{(p)}
\end{Highlighting}
\end{Shaded}

\includegraphics{datacraft_NB_2021_05_12_files/figure-latex/unnamed-chunk-3-1.pdf}

Without retweets

\begin{Shaded}
\begin{Highlighting}[]
\NormalTok{daily_TS_original <-}\StringTok{ }\KeywordTok{data.frame}\NormalTok{(}\KeywordTok{table}\NormalTok{(}\KeywordTok{substr}\NormalTok{(dataset_clean}\OperatorTok{$}\NormalTok{tweet_date[}\KeywordTok{which}\NormalTok{(}\OperatorTok{!}\NormalTok{dataset_clean}\OperatorTok{$}\NormalTok{is_rt)], }\DataTypeTok{start =} \DecValTok{1}\NormalTok{,}\DataTypeTok{stop =} \DecValTok{10}\NormalTok{)))}
\NormalTok{daily_TS_original}\OperatorTok{$}\NormalTok{Var1 <-}\StringTok{ }\KeywordTok{as.Date}\NormalTok{(daily_TS_original}\OperatorTok{$}\NormalTok{Var1)}
\KeywordTok{colnames}\NormalTok{(daily_TS_original) <-}\StringTok{ }\KeywordTok{c}\NormalTok{(}\StringTok{"dates"}\NormalTok{,}\StringTok{"N"}\NormalTok{)}
\NormalTok{p <-}\StringTok{ }\KeywordTok{ggplot}\NormalTok{(daily_TS_original, }\KeywordTok{aes}\NormalTok{( dates, N)) }\OperatorTok{+}
\StringTok{  }\KeywordTok{geom_bar}\NormalTok{(}\DataTypeTok{stat=}\StringTok{"identity"}\NormalTok{, }\DataTypeTok{na.rm =} \OtherTok{TRUE}\NormalTok{)}\OperatorTok{+}
\StringTok{  }\KeywordTok{scale_x_date}\NormalTok{(}\DataTypeTok{labels=}\KeywordTok{date_format}\NormalTok{ (}\StringTok{"%b %y"}\NormalTok{), }\DataTypeTok{breaks=}\KeywordTok{date_breaks}\NormalTok{(}\StringTok{"1 month"}\NormalTok{)) }\OperatorTok{+}\StringTok{ }\KeywordTok{theme_minimal}\NormalTok{()}
\NormalTok{p}
\end{Highlighting}
\end{Shaded}

\includegraphics{datacraft_NB_2021_05_12_files/figure-latex/unnamed-chunk-4-1.pdf}

\hypertarget{without-rt}{%
\section{Without RT :}\label{without-rt}}

daily\_TS\_original \textless-
data.frame(table(substr(dataset\_clean\(tweet_date[which(!dataset_clean\)is\_rt){]},
start = 1,stop = 10)))
daily\_TS\_original\(Var1 <- as.Date(daily_TS_original\)Var1)
colnames(daily\_TS\_original) \textless- c(``dates'',``N'') p \textless-
ggplot(daily\_TS\_original, aes( dates, N)) +
geom\_bar(stat=``identity'', na.rm = TRUE)+
scale\_x\_date(labels=date\_format (``\%b \%y''),
breaks=date\_breaks(``1 month'')) + theme\_minimal() p

\hypertarget{stats-desc}{%
\subsection{Stats desc}\label{stats-desc}}

length(unique(dataset\_clean\(pseudo)) # users uniques length(unique(dataset_clean\)status\_id))
\# tweets uniques, RT compris
length(unique(dataset\_clean\(pseudo[which(!dataset_clean\)is\_rt){]}))
\# Users uniques qui se sont exprimés
length(unique(dataset\_clean\(status_id[which(!dataset_clean\)is\_rt){]}))
\# Tweets uniques sans RT

```

When you save the notebook, an HTML file containing the code and output
will be saved alongside it (click the \emph{Preview} button or press
\emph{Ctrl+Shift+K} to preview the HTML file).

The preview shows you a rendered HTML copy of the contents of the
editor. Consequently, unlike \emph{Knit}, \emph{Preview} does not run
any R code chunks. Instead, the output of the chunk when it was last run
in the editor is displayed.

\end{document}
